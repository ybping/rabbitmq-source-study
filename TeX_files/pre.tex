\chapter{前言}

RabbitMQ实现了高级消息队列协议,是比较常见的消息中间件。平时工作中,RabbitMQ也算是常用的组件了,也因为RMQ出现过多次事件或事故。有些是因为使用姿势不对,有些因为RMQ本身的问题, 但多次出现问题大多数人摸不着头脑,很是头疼; 在网上搜了下发现关于RMQ的源码分析只有寥寥几篇,还都是好几年前的,遂下定决心学习下RMQ的源码,有很大一部分决心来自于陈帅大大的提议。

学习计划主要分为以下几个部分:第二章学习RMQ的启动机制;第三章节学习RMQ的网络层模块; 第四章学习RMQ的存储管理;第五章学习RMQ的AMQP协议实现,主要包括broker、vhost、 exchang、 queue、 messag等;第六章学习下RMQ的高可用和集群实现。

个人能力有限,文中错误之处还请大家斧正;也欢迎大家一起来分析,相互学习进步。